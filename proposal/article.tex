% cs260 project proposal
% limit two pages, due Monday April 17

% layout
\documentclass[11pt, twocolumn, letterpaper]{article}
%\documentclass[10pt, letterpaper]{article}
\usepackage{fullpage}
\frenchspacing
%\usepackage{parskip}

% fonts
\usepackage{charter}
\usepackage{courier}

% other stuff
\usepackage{graphicx}
\usepackage{url}

\usepackage{color}
\newcommand{\XXX}[1]{{\bf\color{red} #1}}

\title{CS260 Project Proposal}
\author{Rob Bowden, David Holland, Eric Lu}
\date{April 17, 2017}

\begin{document}
\maketitle

% we don't have enough space to have an abstract
% or alternatively, the whole thing is an abstract
\begin{abstract}
\end{abstract}

\section{Introduction}

% What kind of system are you interested in?
%
% What property are you aiming to verify?

\section{Goals}

% Give a Coq statement of your capstone theorem. This theorem
% statement need not compile, it may rely on types you haven't defined
% yet, but it should correspond to the property you want to verify.
%
% What are the risks? What are you most worried about in your
% development? If you cannot prove the capstone theorem, what simpler
% theorems are you more likely to be able to prove?

\section{Schedule}

% What is your project schedule? What do you aim to have completed
% each week?
%
% What is your division of labor? Who?s doing what? How can you work
% in parallel?

\section{Future Work}

% What is your (hypothetical) future work? How could future
% generations build on your effort?

%%%%%%%%%%%%%%%%%%%%%%%%%%%%%%%%%%%%%%%%%%%%%%%%%%%%%%%%%%%%

%{
%\footnotesize
%\bibliographystyle{abbrv}
%\bibliography{article}
%}

\end{document}
